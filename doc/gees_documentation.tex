%%%%%%%%%%%%%%%%%%%%%%%%%%%%%%%%%%%%%%%%%%%%%%%%%%%%%%%%%%%%%%%%%%%%%%
%
% Copyright 2012-2015 Arno Mayrhofer
% This program is licensed under the GNU General Public License.
%
% This file is part of Gees.
%
% Gees is free software: you can redistribute it and/or modify
% it under the terms of the GNU General Public License as published by
% the Free Software Foundation, either version 3 of the License, or
% (at your option) any later version.
%
% Gees is distributed in the hope that it will be useful,
%
% MERCHANTABILITY or FITNESS FOR A PARTICULAR PURPOSE.  See the
% GNU General Public License for more details.
%
% You should have received a copy of the GNU General Public License
% along with Gees.  If not, see <http://www.gnu.org/licenses/>.
%
%%%%%%%%%%%%%%%%%%%%%%%%%%%%%%%%%%%%%%%%%%%%%%%%%%%%%%%%%%%%%%%%%%%%%%

\documentclass[a4paper]{article}

\usepackage{fullpage}
\usepackage{amsmath}

\newcommand{\uvec}[1]{\underline{#1}}
\newcommand{\unabla}{\uvec{\nabla}}
\newcommand{\umat}[1]{\uvec{\uvec{#1}}}
\newcommand{\td}{\mbox{d}}

\begin{document}

\title{GEES - GPL Euler equation solver}
\author{Arno Mayrhofer}

\maketitle

\section{Mathematical foundation}
The compressible Euler equations are given in conservative form by
\begin{equation}
\frac{\partial \uvec{u}}{\partial t} + \unabla \cdot \uvec{F}(\uvec{u}) = 0,
\label{e:euler}
\end{equation}
where $\uvec{u} = (\rho, \rho v)^T$ is the state vector with $\rho$ and $v$ being the density and velocity, respectively. As GEES is a 1-D solver, the velocity is given as a scalar. The flux variable is given as
\begin{equation}
\uvec{F}(\uvec{u}) = (u_2, u_2^2/u_1 + p),
\label{e:flux}
\end{equation}
where $p$ is the pressure which is linked to the density via the equation of state, given as
\begin{equation}
p = \frac{c_0^2 \rho_0}{\gamma}\left[ \left(\frac{\rho}{\rho_0} \right)^\gamma - 1\right],
\label{e:eos}
\end{equation}
where $c_0$ is the speed of sound, $\rho_0$ is the reference density and $\gamma$ is a coefficient taken to be 7 in the case of water as fluid.
\\
The flux Jacobian is given as
\begin{equation}
\frac{\td \uvec{F}}{\td \uvec{u}} =
\begin{pmatrix}
0 & 1 \\
-u_2^2/u_1^2 + c_0^2 (u_1/\rho_0)^{\gamma-1} & 2 u_2/u_1
\end{pmatrix}.
\label{e:jacobian}
\end{equation}
The eigenvalues of this matrix are given as
\begin{equation}
a^\pm = u_2/u_1 \pm c_0(u_1/\rho_0)^{(\gamma-1)/2}.
\label{e:ev}
\end{equation}

\end{document}
